% Chapter 2, 8, 10, 11, 14

\documentclass[12pt]{article}
\usepackage[a4paper, margin=0.75in]{geometry}
\usepackage{amsmath, amssymb}
\pagestyle{empty}

% Commands
\newcommand{\cosec}{\mathop{\mathrm{cosec}}}
\everymath{\displaystyle}

\begin{document}

\begin{center}
    {\LARGE \textbf{Higher Mathematics Model Test}}\\[10pt]
    \textbf{Time: 2 hours 35 minutes \hfill Marks: 50}
\end{center}

\vspace{10pt}
\noindent\textbf{Instructions:} 
\begin{itemize}
    \item Answer \textbf{five} questions from the Creative Essay-Type section, taking at least \textbf{one} from each part.
    \item Answer \textbf{any five} questions from the Short Answer section.
\end{itemize}

\vspace{5pt}
\section*{Creative Essay-Type Questions}

\subsection*{Part A: Algebra}

\begin{enumerate}
    \item Let \( A = (1 - 2x + x^2)^3 \) and \( B = \left(k + \dfrac{3}{x}\right)^5 \).
    \begin{enumerate}
        \item Expand \( (2 - 3x)^4 \) using Pascal’s triangle.
        \item Expand \( A \) using the binomial theorem.
        \item Given that the coefficient of \( \dfrac{1}{x^4} \) in \( B \) is \( -810 \), find the value of \( k \).
    \end{enumerate}

    \item Let \( P(x) = x^4 + 7x^3 + 17x^2 + 17x + 6 \) and \( Q(x, y, z) = \dfrac{1}{x^3} + \dfrac{1}{8y^3} + \dfrac{1}{64z^3} \).
    \begin{enumerate}
        \item Show that \( (x + 1) \) is a factor of \( F(x) = x^3 - x^2 + 5x + 7 \).
        \item Factorize \( P(x) \).
        \item If \( Q(x, y, z) = \dfrac{3}{8xyz} \), show that \( 4yz + 2zx + xy = 0 \) or \( x = 2y = 4z \).
    \end{enumerate}

    \item Let \( P = a^3(b - c) + b^3(c - a) + c^3(a - b) \) and \( Q = \dfrac{x}{(x - 1)(x^2 + 4)} \).
    \begin{enumerate}
        \item Show that \( F(x, y, z) = xy + yz + zx \) is a symmetric expression.
        \item Factorize \( P \).
        \item Decompose \( Q \) into partial fractions.
    \end{enumerate}
\end{enumerate}

\subsection*{Part B: Geometry}

\begin{enumerate}
    \setcounter{enumi}{3}
    \item Consider the lines:
    \begin{align*}
        &\text{(i)}\quad y = 3x - 10, &&\text{intersects the } x\text{-axis at point } A; \\
        &\text{(ii)}\quad 2x - y = 4, &&\text{intersects the } y\text{-axis at point } B; \\
        &\text{(iii)}\quad x - 2y + 10 = 0.
    \end{align*}
    Lines (i) and (ii) intersect at point \( C \).
    \begin{enumerate}
        \item Find the slope of line (iii).
        \item Show that lines (i), (ii), and (iii) are concurrent.
        \item Calculate the area of triangle \( \triangle ABC \).
    \end{enumerate}

    \item Let \( A = (-4, 13) \), \( B = (8, 8) \), \( C = (13, -4) \), and \( D = (1, 1) \) be the vertices of quadrilateral \( ABCD \).
    \begin{enumerate}
        \item Find the angle between line \( BD \) and the \( x \)-axis.
        \item Determine the nature of the quadrilateral \( ABCD \).
        \item Find the area of the portion of quadrilateral \( ABCD \) that forms a triangle with the \( x \)-axis.
    \end{enumerate}
\end{enumerate}

\subsection*{Part C: Trigonometry and Probability}

\begin{enumerate}
    \setcounter{enumi}{5}
    \item Let \( a = \cot\theta \) and \( b = \cosec\theta \).
    \begin{enumerate}
        \item If \( 4\theta = \pi \), find the value of \( a^2 - b \).
        \item Given \( a + b = x \), show that \( \cos\theta = \dfrac{x^2 - 1}{x^2 + 1} \).
        \item If \( 3(a^2 + b^2) = 5 \), find the values of \( \theta \) where \( 0 < \theta < 2\pi \).
    \end{enumerate}

    \item The probability of Rahim traveling from Dhaka to Chattogram by bus is \( \dfrac{3}{5} \), and by train is \( \dfrac{1}{5} \). The probability of traveling from Chattogram to Cox’s Bazar by bus is \( \dfrac{6}{13} \), and by airplane is \( \dfrac{3}{26} \).
    \begin{enumerate}
        \item What is the probability of getting a prime number when rolling a standard dice?
        \item Draw a probability tree diagram for the above travel scenario.
        \item Find the probability that Rahim travels to Chattogram \textbf{not} by train and then travels to Cox’s Bazar by \textbf{neither} bus nor airplane.
    \end{enumerate}
\end{enumerate}

\section*{Short Answer Questions}

Answer any five questions.

\begin{enumerate}
    \item Find the coefficient of \( x^2 \) in the expansion of \( (1 - x)^5 \).
    \item Find the remainder when \( x^3 - 2x^2 + x - 7 \) is divided by \( x + 3 \).
    \item Decompose \( \dfrac{1}{x^2 + 5x + 6} \) into partial fractions.
    \item Given a point \( P(x, y) \), the distance to the \( y \)-axis is equal to the distance from \( P \) to the point \( Q(3, 2) \). Prove that \( y^2 - 4y - 6x + 13 = 0 \).
    \item Show that if \( A(a, b) \), \( B(b, a) \), and \( C\left(\dfrac{1}{a}, \dfrac{1}{b}\right) \) are collinear, then \( a + b = 0 \).
    \item Prove that
    \[
        \cos\left(\dfrac{17\pi}{10}\right) + \cos\left(\dfrac{13\pi}{10}\right) + \cos\left(\dfrac{9\pi}{10}\right) + \cos\left(\dfrac{\pi}{10}\right) = 0.
    \]
    \item In a lottery of 500 tickets with 30 prizes, all tickets are sold. Rahim owns 5 tickets. Prizes are drawn in reverse order—from the 30th to the 1st. What is the probability that Rahim wins the 3rd prize, given that none of his tickets won any of the 27 previous prizes?
\end{enumerate}

\end{document}
